\documentclass[12pt,a4paper]{report}
\usepackage[utf8]{inputenc}
\usepackage[french]{babel}
\usepackage[T1]{fontenc}
\usepackage{amsmath}
\usepackage{amsfonts}
\usepackage{tcolorbox}
\newtcolorbox{mybox}[1]{colback=blue!5!white,colframe=blue!60!black,fonttitle=\bfseries,title=#1}

\newtcolorbox{Cas1}[1]{colback=green!5!white,colframe=green!60!black,fonttitle=\bfseries,title=#1}

\newtcolorbox{Cas2}[1]{colback=yellow!5!white,colframe=yellow!60!black,fonttitle=\bfseries,title=#1}


\newtcolorbox{Cadre}{colback=red!5!white,colframe=red!75!black}

\usepackage[Sonny]{fncychap}
\usepackage{listings}
\usepackage{amssymb}
\usepackage{graphicx}
\usepackage[left=2cm,right=2cm,top=2cm,bottom=2cm]{geometry}
\author{ADOLPHE Benjamin-BERJOLA Matthias}
\title{Multiplication Russe}
\begin{document}


\title{Multiplication Russe}
\author{ADOLPHE Benjamin-BERJOLA Matthias}


\maketitle 

\newpage
\chapter{Introduction}
\begin{large}
\begin{flushleft}
Dans le cadre de l'UE calculabilité et complexité nous avions dû réaliser plusieurs tâches sur un algorithme comprenant un contrat pré-conditions/post-conditions clair, et incluant au moins une boucle tant-que.Nous avons choisis de travailler sur l'algorithme représentant la multiplication russe et d'effectuer les tâches suivante sur ce dernier: 

\begin{itemize}
\item[•] Écrire son code Dafny
\medskip

\item[•] Prouver sa terminaison : fonctions de rang et éventuels invariants seront présentées et prouvés
\medskip 

\item[•] Montrer la  correction partielle : les invariants seront présentées et prouvés, ainsi que les post-conditions.
\medskip

\item[•] Déterminer sa complexité en temps qui dans le pire des cas sera justifiée et possiblement validée expérimentalement

\medskip

Nous allons donc exposé nos travaux dans ce rapport en prenant soin de suivre l'ordre décrit ci-dessus
\end{itemize}
\end{flushleft}
\end{large}

%*********************************
%*******Code Dafny**************											   
%*********************************                                                
%*********************************
\chapter{Code en Dafny}
\includegraphics[scale=0.45]{algoDafny} 
\includegraphics[scale=0.5]{algoRusse} 



%*********************************
%*******Correction totale******											   
%*********************************                                                
%*********************************

\chapter{Correction totale}
\section{Correction partielle}
\begin{mybox}{Méthode de détermination de la terminaison}
On commence par déterminer l'invariant de boucle en posant les cas de base \\
et de récursivité.On détermine ensuite une fonction de rang et on prouve que celle-ci est valide.

\end{mybox}
\begin{flushleft}
Invariant en (*) $  a\geq 0 $  : \\
\begin{itemize}
\item Cas Inductif :  $x$ est affect\'{e} \`{a} $a$ or $x$ est un entier naturel ainsi par typage $a\ge 0$.
\item Cas Récursif : nous supposons que l’invariant  $a\ge 0$ est vrai, montrons alors: $ a' \geq 0 $ \medskip
\begin{itemize}
\item 1er cas :  est pair. Nous savons que $\left[ \begin{array}{c}(a\ge 0\wedge a>0\wedge \\
   a '=(\frac{a}{2})\wedge a\%2=0 \wedge \\
b '=2*b\wedge r '=r)\end{array}
\right]$. Ainsi nous avons $a\ge 0\Leftrightarrow \frac{a}{2}\ge \frac{0}{2}\Leftrightarrow a ' \ge 0$.\\
\item 2ème cas : a est impair: Nous savons que $\left[ \begin{array}{c}(a\ge 0\wedge a>0\wedge \\
   r '= r+b\wedge b'=b  \wedge \\
a '=a+1\wedge a\%2=1\end{array}
\right]$. Ainsi nous avons  $a\ge 0$ d’après le test de boucle or \medskip $  a\geq 0  \Leftrightarrow a-1 \ge 0-1 \Leftrightarrow a' \ge 0$ .
\bigskip
\end{itemize}
\end{itemize}
\begin{Cas1}{Conslusion}
 Dans les deux cas, nous avons bien montré que  $ a' \ge 0 $. 
 $a \geq 0$ est donc bien un invariant de boucle en $(*)$ .\\
\end{Cas1}
\end{flushleft}

\begin{flushleft}
Invariant $(*) r+a*b = x*y$ :\\
\begin{itemize}
\item Cas Inductif : 0   est affecté à r, x  est affecté à a  et y   est affecté à b  ainsi $ r +a*b = 0 + a*b = x*y $. On a donc bien $ r + a*b = x*y$ .
\item Cas Récursif : nous supposons que l’invariant  $ r + a*b = x*y$ est vrai, montrons alors que  $ r' + a'*b' = x'*y'$ est vrai :
\begin{itemize}
\item 1er cas : a est pair:  Nous savons que $\left[ \begin{array}{c}(r + a*b = x*y\wedge a>0\wedge \\
a\%2=0\wedge a'= \frac{a}{2} \wedge r'=r \wedge\\
b'=2*b\wedge x'=x \wedge y'=y\end{array}
\right]$  Ainsi nous avons $ r' + a'*b' = r' + \frac{a}{2} * 2 * b  = r+ a*b$ or d’après
 invariant $ r+a*b = x*y $ et  $ x*y = x'*y'$ Nous avons donc bien\medskip $ r'+a'*b' = x'*y'$
\item 2ème cas : a est impair. Nous savons que $\left[ \begin{array}{c}(r + a*b = x*y\wedge a>0\wedge \\
a\%2=1\wedge a'= a-1\wedge b'=b \wedge\\
r'=r+b\wedge b'=b\wedge x'=x \wedge y'=y\end{array}
\right]$ Ainsi nous avons $r'+ a'*b'=r+b+(a-1)*b = r+a*b + b-b = r+a*b  $ or d’après l’invariant nous avons  $ r+a*b = x*y $et $ x*y = x'*y' $ Nous avons donc bien $ r'+a'*b' = x'*y'$
\end{itemize}
\end{itemize}
\begin{Cas1}{Conclusion}
 Dans les deux cas, nous avons montré que $ r'+a'*b'= x'*y' $.
$ r+a*b=x*y $est donc un invariant de boucle en $(*) $.
\end{Cas1}
\end{flushleft}

\section{Terminaison}

%*********************************
%*******Complexité**************											   
%*********************************                                                
%*********************************

\chapter{Complexité}

\end{document}